Our program can make the robot explore most of the map, the issues still present are the fact that when we compute a path we stop the robot to avoid it going into obstacles.
That can be annoying because the robot will stop every 5 seconds since we compute the path every 5 seconds.

Another issue is the way that we select the frontier, indeed we select the closest frontier from the robot, sometimes the closest frontier is behind a wall.
When we are in that case an other issue appears, the fact that the A* algorithm will sometimes find a path that goes through a wall (When the wall is not yet fully discovered), leading the robot to stay stuck.
A way of correcting that issue would have been to create a map on which the obstacles are expanded which would have make it easier to find a correct path.

Under we can see a map almost fully explored by the robot.

\FloatBarrier
\begin{figure}
    \centering\includegraphics[width=0.5\textwidth]{map.png}
    \label{fig:map}
    \caption{Map almost fully explored}
\end{figure}
\FloatBarrier

Under we can see the robot while exploring a map.

\FloatBarrier
\begin{figure}
    \centering\includegraphics[width=\textwidth]{full_screen.png}
    \label{fig:full_screen}
    \caption{Full screen}
\end{figure}
\FloatBarrier
