\section{Code structure}

Our program is divided in three main modules, the mapping module, the planning module and the controller module.
The role of the first one is to build the map of the environment of the robot using the echoes of the lasers.
The role of the second one is to plan the goal of the robot and the path that the robot must follow in order to explore the world.
The role of the controller module is to make the robot move towards its goal while avoiding the obstacles in its way.

To communicate with the MRDS server we created a class named 'Robot', it is used as an interface to send and receive informations to and from the MRDS server easily.
The received informations are directly adapted to our needs and stored using our 'Position' and 'Laser' datastructures in this class.

\section{How to run our program}

For this assignment we used Python 3 as programming language, to run the program run the 'mapper.sh' script with the following syntax:
\\
\begin{lstlisting}[language=bash, basicstyle=\small]
mapper.sh url x_1 y_1 x_2 y_2 show_gui
\end{lstlisting}

The parameters of the type $x\_1$ correspond to the lower left position and upper right position respectively.

For more details about those parameters you can read the following:
\\
The 'mapper.sh' script above launches our program with the 6 parameters.

If you'd rather not use the script from above you can directly launch the 'main.py' python script.
The usage of the python program is the following:
\\
\begin{lstlisting}[language=bash, basicstyle=\small]
usage:python3 main.py [-h] url lower_left_pos_x 
    lower_left_pos_y upper_right_pos_x upper_right_pos_y show_gui

positional arguments:
  url                The url of the MRDS server with the port, 
                     the format is url:port
  lower_left_pos_x   The X coordinate of the lower left position.
  lower_left_pos_y   The Y coordinate of the lower left position.
  upper_right_pos_x  The X coordinate of the upper right position.
  upper_right_pos_y  The Y coordinate of the upper right position.
  show_gui           1 if you want to show the GUI, 0 otherwise.

optional arguments:
  -h, --help         show this help message and exit
\end{lstlisting}
