To optimise the performances we choose to use multiprocessing.

\section{Cartographer}

We've put the cartographer in its own subprocess, it updates the map every 0.1 seconds.
In that way the map is always efficiently updated.

The map is then 'sent' to the main program through a Queue (object of the multiprocessing python library).
In that way the main program can recuperate the last version of the map at any time needed.

\section{Show map}

Since we usually print out many informations on our map: the frontiers, goal point and forces applied to the robot, the update of the graphical window takes some time.
It came to a point where the update of the graphical window was the longest part in our program.
We could not afford to lose that much time so we created a subprocess for the ShowMap.

The graphical window is updated every half a second.

\section{Communication with MRDS}

To communicate with MRDS we were using the Robot class as an interface, the only thing that it did was send the request.
Eventually we ran into a case where more than one process tried to access to the robot position through the Robot object simultaneously.
In that case we had an error message from the MRDS server and our program crashed.

To fix this situation we implemented a delay system in the Robot class.
When we access to the Robot interface to request the position or the lasers if a request has already been made in the last 0.1 seconds it will return the result of that last request.
By doing that we are able to use the Robot interface with as many processes as we want.

