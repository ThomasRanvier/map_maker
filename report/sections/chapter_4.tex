\section{Controller structure}

Some parts of our controller are deliberative and others are reactive, this is how they are separated:

\begin{itemize}
    \item[$-$] The path planning is deliberative, indeed it creates a path from the map created by the Cartographer.
        It uses a frontier detector, a goal planner and the A* algorithm for building the path.
    \item[$-$] The path tracking however is reactive, indeed it directly computes the forces to apply to the robot to make it follow the path while avoiding the obstacles.
\end{itemize}

\section{Delegations of the tasks}

We separated the different tasks in that way:

\begin{itemize}
    \item[$-$] Valentin handled most of the mapping part, and the optimisation of all the code by creating subprocesses and making them communicate with each other in a good way.
    \item[$-$] Thomas handled most of frontier detection, goal planning, path planning and potential fields to make the robot find a new goal, build a path and follow it while avoiding the obstacles.
\end{itemize}

\section{Inaccessible frontiers}

At this point the robot was able to explore and discover its environment by itself, but there were some details that we could improve.

Sometimes a frontier is found in an inaccessible place, then the robot will go to the closest frontier it can and the closest frontier will always remain an inaccessible one, the robot will then stay stuck.

To fix this issue we had to find a way to detect when the robot stays stuck and then delete the frontier that it is trying to reach.

We created a new process called 'frontiers\_limiter' that would be executed in the back ground, it runs every second and memorises the actual position of the robot.
It communicates with the goal planner which send to it the last closest frontier as soon as it is computed, the frontiers limiter sends to the goal planner a list of all the cells to ignore when it is building the frontiers.

It only memorise the last 20 positions of the robot, which make it delete every position older than 20 seconds.
Once it has memorised 20 positions it computes $\Delta x$ and $\Delta y$ and if those $\Delta$s are bellow 3.5 meters for both of them the robot is considered as stuck, because it means that the robot stayed in a square of 3.5 meters by 3.5 meters for the last 20 seconds.

Also if the robot is detected as being immobile for 5 consecutive seconds the frontier is deleted.

If the robot is detected as being stuck the closest frontier will be deleted in this way: we go through all the points in the frontier and add all the cells in a radius of 6 cells around each point to the list of cells to ignore, then it sends the updated list to the goal planner.
In that way the goal planner will ignore the cells in the list when it is building the frontiers in the future.

